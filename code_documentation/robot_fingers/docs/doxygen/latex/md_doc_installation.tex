We provide a Singularity image with all required dependencies to build and run the software. See \hyperlink{md_doc_singularity}{About Singularity}.

You can, of course, also use the package without Singularity. In this case you need to install all dependencies locally, though.

\subsection*{Get the Source }

{\ttfamily robot\+\_\+fingers} depends on several other of our packages which are organized in separate repositories. We therefore use a workspace management tool called \href{https://pypi.org/project/treep/}{\tt treep} which allows easy cloning of multi-\/repository projects.

treep can be installed via pip\+: \begin{DoxyVerb}pip install treep
\end{DoxyVerb}


Clone the treep configuration containing the \char`\"{}\+R\+O\+B\+O\+T\+\_\+\+F\+I\+N\+G\+E\+R\+S\char`\"{} project\+: \begin{DoxyVerb}git clone git@github.com:machines-in-motion/treep_machines_in_motion.git
\end{DoxyVerb}


{\bfseries Note\+:} treep searches for a configuration directory from the current working directory upwards. So you can use treep in the directory in which you invoked the {\ttfamily git clone} command above or any subdirectory.

Now clone the project\+: \begin{DoxyVerb}treep --clone ROBOT_FINGERS
\end{DoxyVerb}


{\bfseries Important\+:} treep uses S\+SH to clone from github. So for the above command to work, you need a github account with a registered S\+SH key. Further this key needs to work without asking for a password everytime. To achieve this, run \begin{DoxyVerb}ssh-add
\end{DoxyVerb}


first.

\subsection*{Build }

\subsubsection*{With Singularity}

Go to the root directory of your workspace (the one containing the \char`\"{}src\char`\"{} folder) and run the container in shell mode (see \hyperlink{md_doc_singularity}{About Singularity})\+: \begin{DoxyVerb}singularity shell -e --no-home -B $(pwd) path/to/image.sif
\end{DoxyVerb}


The current working directory gets automatically mounted into the container so you can edit all the files from outside the container using your preferred editor or I\+DE and all changes will directly be visible inside the container. Vise versa modifications done from inside the container will modify the files on the host system!

Inside the container first set up the environment\+: \begin{DoxyVerb}Singularity> source /setup.bash
\end{DoxyVerb}


This will source the R\+OS {\ttfamily setup.\+bash} and create an alias {\ttfamily catbuild} for catkin that already contains the argument to set the Python executable to python3 (needed to get Python 3 bindings).

Now you can build by using this alias\+: \begin{DoxyVerb}Singularity> catbuild
\end{DoxyVerb}


\subsubsection*{Without Singularity}

To build, cd into the {\ttfamily workspace} directory and build with \begin{DoxyVerb}catkin build
\end{DoxyVerb}


\subsubsection*{Python Bindings}

With the above command Python bindings will be build for the default python version of your system (see {\ttfamily python -\/-\/version}). If you want to use a different version (e.\+g. python3), you can specify as follows\+: \begin{DoxyVerb}catkin build -DPYTHON_EXECUTABLE=/usr/bin/python3
\end{DoxyVerb}


Note that all our scripts are implemented for Python 3, so if your default version is not already 3, you need to specify this in order to use these scripts. 