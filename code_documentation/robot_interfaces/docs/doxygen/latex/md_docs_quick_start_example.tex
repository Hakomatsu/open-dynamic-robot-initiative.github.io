Sending actions to and getting observations from the robot is very easy. See the following example, using the Tri\+Finger robot, that simply sends a constant position command.

\begin{DoxyNote}{Note}
This example shows only the frontend part of a multi-\/process setup. The backend for the actual robot needs to be run in a separate process.
\end{DoxyNote}
\subsection*{Python}


\begin{DoxyCode}
\textcolor{keyword}{import} robot\_interfaces

robot\_data = robot\_interfaces.trifinger.MultiProcessData(
    \textcolor{stringliteral}{"trifinger"}, \textcolor{keyword}{False})
frontend = robot\_interfaces.trifinger.Frontend(robot\_data)

position = [
     0.0,  \textcolor{comment}{# Finger 1, Upper Joint}
    -0.9,  \textcolor{comment}{# Finger 1, Middle Joint}
    -1.7,  \textcolor{comment}{# Finger 1, Lower Joint}
     0.0,  \textcolor{comment}{# Finger 2, Upper Joint}
    -0.9,  \textcolor{comment}{# Finger 2, Middle Joint}
    -1.7,  \textcolor{comment}{# Finger 2, Lower Joint}
     0.0,  \textcolor{comment}{# Finger 3, Upper Joint}
    -0.9,  \textcolor{comment}{# Finger 3, Middle Joint}
    -1.7,  \textcolor{comment}{# Finger 3, Lower Joint}
]

\textcolor{keywordflow}{while} \textcolor{keyword}{True}:
    \textcolor{comment}{# construct an action with a position command}
    action = robot\_interfaces.trifinger.Action(position=position)
    \textcolor{comment}{# send the action to the robot (will be applied in time step t)}
    t = frontend.append\_desired\_action(action)
    \textcolor{comment}{# wait until time step t and get observation}
    observation = frontend.get\_observation(t)

    print(\textcolor{stringliteral}{"Observed Position: \{\}"}.format(observation.position))
\end{DoxyCode}


\subsection*{C++}


\begin{DoxyCode}
\textcolor{preprocessor}{#include <robot\_interfaces/finger\_types.hpp>}

\textcolor{comment}{// Some convenience typedefs to make the code below more compact}
\textcolor{keyword}{typedef} \hyperlink{classrobot__interfaces_1_1MultiProcessRobotData}{robot\_interfaces::TriFingerTypes::MultiProcessData}
       RobotData;
\textcolor{keyword}{typedef} \hyperlink{classrobot__interfaces_1_1RobotFrontend}{robot\_interfaces::TriFingerTypes::Frontend} RobotFrontend;
\textcolor{keyword}{typedef} \hyperlink{structrobot__interfaces_1_1NJointAction}{robot\_interfaces::TriFingerTypes::Action} Action;

\textcolor{keywordtype}{int} main()
\{
    \textcolor{keyword}{auto} robot\_data = std::make\_shared<RobotData>(\textcolor{stringliteral}{"trifinger"}, \textcolor{keyword}{false});
    \textcolor{keyword}{auto} frontend = RobotFrontend(robot\_data);

    Action::Vector position;  \textcolor{comment}{// <- this is an "Eigen::Vector9d"}
    position <<  0.0,  \textcolor{comment}{// Finger 1, Upper Joint}
                -0.9,  \textcolor{comment}{// Finger 1, Middle Joint}
                -1.7,  \textcolor{comment}{// Finger 1, Lower Joint}
                 0.0,  \textcolor{comment}{// Finger 2, Upper Joint}
                -0.9,  \textcolor{comment}{// Finger 2, Middle Joint}
                -1.7,  \textcolor{comment}{// Finger 2, Lower Joint}
                 0.0,  \textcolor{comment}{// Finger 3, Upper Joint}
                -0.9,  \textcolor{comment}{// Finger 3, Middle Joint}
                -1.7;  \textcolor{comment}{// Finger 3, Lower Joint}

    \textcolor{keywordflow}{while} (\textcolor{keyword}{true})
    \{
        \textcolor{comment}{// construct an action with a position command}
        Action action = Action::Position(position);
        \textcolor{comment}{// send the action to the robot (will be applied in time step t)}
        \textcolor{keyword}{auto} t = frontend.append\_desired\_action(action);
        \textcolor{comment}{// wait until time step t and get observation}
        \textcolor{keyword}{auto} observation = frontend.get\_observation(t);

        std::cout << \textcolor{stringliteral}{"Observed Position: "}
                  << observation.position
                  << std::endl;
    \}

    \textcolor{keywordflow}{return} 0;
\}
\end{DoxyCode}


When using C++ you need to add the package {\ttfamily robot\+\_\+interfaces} as build dependency to your package.

\subsection*{More Examples}

For more examples, see the \href{https://github.com/open-dynamic-robot-initiative/robot_interfaces/tree/master/demos}{\tt C++ demos of the {\ttfamily robot\+\_\+interfaces} package} and the \href{https://github.com/open-dynamic-robot-initiative/robot_fingers/tree/master/demos}{\tt Python demos in the {\ttfamily robot\+\_\+fingers} package}. 